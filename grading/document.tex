\documentclass{article}
\author{Florian Hansen \and Michael Frank}
\title{Bewertungsbericht für Gruppe 9: Next-Intent Vulnerability}

\usepackage{setspace}
\usepackage[ngerman]{babel}
\usepackage{csquotes}

\MakeOuterQuote{"}
\setlength{\parindent}{0cm}
\setlength{\parskip}{0.15cm}
\onehalfspacing

\begin{document}
  \maketitle

  Die Gruppe 9 hat sich mit der Arbeit von \cite{niv} beschäftigt, welche die
  Schwachstellen (NIVs) in öffentlichen und privaten \texttt{Intents} des
  Android-Betriebssystems behandelt. Kurz zusammengefasst ist der Text
  verständlich, wir sehen jedoch in seiner Form sowie im Inhalt
  Verbesserungsbedarf. Die Ausarbeitung der Gruppe 9 erhält von uns 
  \textbf{7 von 10 Punkten}. Im Folgenden begründen wir die Bewertung.

  Im gesamten Text findet man eine Mischung von englischen und deutschen
  Fachbegriffen, wie beispielsweise die Formulierung \textit{"Diese spezifische
  Vulnerability betrifft allerdings [...]"}. Die Einhaltung einer Sprache
  erscheint uns als angemessener und wissenschaftlicher. Zudem haben wir
  diverse Fehler in der Grammatik feststellen können. Auch dies trägt nicht
  gerade zur Lesbarkeit bei. Ein Beispiel hierfür ist die Formulierung aus
  Abschnitt 2 \textit{"So könnte es z.B. passieren, dass ein/e Angreifer*in beim
  Betrachten der Appstruktur eine öffentlich zugängliche Komponente
  detektiert, die Variablen an eine private Komponente übergibt"}. Wie man in
  dem Beispiel und in anderen Abschnitten sieht, werden oft umgangssprachliche
  Gebilde wie \textit{"kleine Applikationen"} verwendet. "Klein" kann in
  diesem Kontext alles bedeuten. Ob damit die Anzahl der Zeilen der Codebasis,
  die Anzahl an Schwachstellen, die Speichergröße, die Popularität oder
  die Einfachheit einer Anwendung gemeint ist, ist nicht klar und muss vermutet
  werden. \textbf{Aufgrund der genannten Kritikpunkte ziehen wir der Gruppe einen 
halben Punkt ab.}

  Der Inhalt des Textes ist grundsätzlich verständlich, jedoch fühlt man sich an
  einigen Stellen etwas verlassen. Es werden diverse Fachwörter verwendet, ohne
  sie auch nur im Ansatz zu erläutern. So ist der Satz \textit{"Eine dieser
  Sicherheitslücken ist die Next-Intent Vulnerability, welche die Kommunikation
  zwischen Android Applikationen als Angriffspfad verwendet."} unverständlich,
  da der Leser nicht genau weiß, was "Angriffspfad" bedeutet. Auch wird dieser
  Begriff in den folgenden Absätzen nicht erläutert, sodass die Schwachstelle
  nicht richtig verstanden werden kann. Im selben Zusammenhang werden auch
  Begriffe wie \textit{"öffentliche Activities"} verwendet, ohne diese zu
  behandeln. Wie dann die Kommunikation und die damit verbundene Schwachstelle
  funktioniert, wird hier nicht klar. Es wird auch häufig auf die
  \textit{"Struktur einer Applikation"} referenziert und damit versucht,
  Schwachstellen zu erläutern. Jedoch wird dies nirgends definiert bzw.
  erläutert, sodass auch hier vermutet werden muss, um was es sich handelt.
  Auch im vierten Abschnitt "Auswertung" hätten wir uns etwas mehr Tiefe
  gewünscht. Es wird zum Beispiel nicht ganz klar, wie die Zahl "100\%" zustande
  gekommen ist. Es ist bspw. interessant, ob es sich im Testlauf um eine einzelne
  Gruppe handelte, welche die manuellen und automatisierten Tests durchführten,
  oder ob zwei voneinander unabhängige Gruppen gewählt wurden. Die einzelne
  Gruppe kann unter Umständen von den vorherigen Tests beeinflusst worden sein.
  Auch wenn die Tiefe des Inhalts darunter leidet, lernt man dennoch einiges
  über die Wichtigkeit beim Umgang mit \texttt{Intents} und Werkzeugen, die
  Schwachstellen detektieren.  \textbf{Wir ziehen der Gruppe ein und halb Punkte
  aufgrund von fehlenden Definitionen und Erläuterungen ab, die den Inhalt
  unverständlich und oberflächlich gestalten.}

  Als nächstes sind uns fehlende Quellenangaben aufgefallen. Zu Beginn der
  Arbeit werden alle Quellen einmal genannt. In den Folgeabschnitten ist jedoch
  nicht mehr ganz klar, auf welche Quelle sich gerade bezogen wird. Man könnte
  bspw. vermuten, dass sich die restlichen Abschnitte fälschlicherweise auf die
  zuletzt genannte Quelle beziehen. Gerade in dem Abschnitt 4 "Auswertung"
  könnte man fälschlicherweise annehmen, dass es sich dabei um die Ergebnisse der 
 Gruppe selbst und nicht einer ihrer Quellen handelt. Aufgrund des Kontextes war uns 
 bewusst, dass sich die Kapitel auf die von Ihnen bearbeitete Ausarbeitung beziehen, 
 dennoch wären Quellenangaben auch in Folgeabschnitten, bspw. am Ende eines Absatzes,
 angebracht gewesen. \textbf{Wir ziehen der Gruppe dafür einen Punkt ab, da hauptsächlich 
nur zum Beginn (1. Kontext) Quellen genannt werden.}

 Die beschriebene Planung für die praktische Demonstration der Sicherheitslücke
 erachten wir als gut und sinnvoll, wir haben diesem Ansatz nichts auszusetzen.

 \pagebreak

 Die Bewertungskategorien mit Gewichtungen und Abzügen waren wie folgt:

 \begin{itemize}
   \item Verständlichkeit (50\%) - \textbf{3.5/5}
   \item Methodik und Tiefe (25\%) - \textbf{1/2.5}
   \item Arbeitsplanung (25\%) - \textbf{2.5/2.5}
 \end{itemize}

\end{document}
