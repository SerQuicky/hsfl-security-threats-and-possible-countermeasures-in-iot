\section{Durchführung des Experiments}
Nachdem der Testaufbau besprochen wurde, gilt es nun in die Rolle des Angreifers
zu schlüpfen und mögliche Angriffe zu diskutieren und auszuführen. Außerdem soll
besprochen werden, wie die Angriffe hätten verhindert werden können, sodass die
Arbeit von~\cite{paper} Anwendung findet. Wie in
Abbildung~\ref{fig:testing-setup} dargestellt, werden die Komponenten zunächst
aufgebaut. Folgende Netzwerkteilnehmer werden konfiguriert:

\begin{itemize}
  \item \textbf{Alice} erhält die IP-Adresse 192.168.0.168 und stellt einen
    MQTT-Client dar. Es werden bestimmte Topics abonniert, sodass dieses Gerät
    benachrichtigt wird, falls sich bestimmte Werte im System ändern.
  \item \textbf{Bob} erhält die IP-Adresse 192.168.0.115 und stellt den
    MQTT-Broker dar. Dieser empfängt Daten von Sensoren und speichert diese
    unter einem bestimmten Topic ab. Abonnenten des Topics werden dann von ihm
    benachrichtigt. Als Anwendung wird \textit{Aedes} verwendet.
  \item Der \textbf{Sicherheitskamera} (Security Camera) wird die IP-Adresse
    192.168.0.178 zugewiesen und dient in dem Aufbau als Sensor. Es werden
    periodisch Bildaufnahmen erzeugt und an Bob gesendet.
  \item \textbf{Eve} stellt den Angreifer dar und erhält die IP-Adresse
    192.168.0.171.
  \item Der \textbf{Router} stellt mit der IP-Adresse 192.168.0.1 das
    Standard-Gateway des Netzwerks dar.
\end{itemize}

\subsection{Eavesdropping}
Ein entscheidender Grund, warum HTTPS als Ergänzung zu HTTP entwickelt wurde,
ist das Senden und Empfangen von sensiblen Informationen wie Passwörter und
anderen benutzerspezifischen Eigenschaften. Dies ist auch bei MQTT bzw. MQTTS
der Fall. Deshalb wird in diesem Abschnitt untersucht, wie sicher das Framework
\textit{pi-aREST} diesbezüglich ist. Als erster Punkt kann genannt werden, dass
es nicht möglich ist, den Kommunikationsport einzustellen. Dieser ist fest in
das Framework eingearbeitet und lautet 1883, was auf einen unverschlüsselten
Kanal hinweist. Da die Ports jedoch vom Serveradministrator individuell
verwendet werden können, muss nun untersucht werden, ob die Kommunikation
tatsächlich unverschlüsselt verläuft. Hierfür wird auf \textit{Eve} das
Netzwerkanalyse-Tool \textit{Wireshark} installiert und ausgeführt. Zudem werden
die Dienste von der Sicherheitskamera und Bob gestartet, sodass diese
miteinander kommunizieren und Daten kontinuierlich ausgetauscht werden. Werden
Pakete von einem Netzwerkteilnehmer zu einem anderen gesendet, sind diese an
jedem Knotenpunkt im Netzwerk abgreifbar. Jeder Knotenpunkt entscheidet dann
anhand der Paket-Header, ob das jeweilige Paket für ihn bestimmt ist. Dadurch,
dass in der Tat keine Verschlüsselung der Daten erfolgt, können mithilfe von
\textit{Wireshark} Bildaufnahmen ausgespäht werden, wie
Abbildung~\ref{fig:wireshark} zeigt.
