\section{Fazit}
Wir haben besprochen, welche Protokolle in der IoT verwendet werden, um Daten
miteinander kontinuierlich auszutauschen und dass Sicherheitsanforderungen
abhängig von ihrer Domäne sind. So sind Anforderungen an die Automobil-Industrie
andere als die der Spielzeug-Domäne. Eine detaillierte Betrachtung des Kontexts
einer IoT-Anwendung ist demnach essentiell für das Etablieren von IT-Sicherheit.
Mit diesem Wissen haben wir uns ein Projekt bzw. Framework ausgesucht, welches
bekannte Protokolle wie MQTT implementiert und einen Datenaustausch zwischen
IoT-Geräten ermöglicht. Mithilfe dieses Frameworks wurde eine Sicherheitskamera
entwickelt und anschließend auf IT-Sicherheit überprüft. Dabei wurden ebenfalls
die verwendeten Angriffstechniken besprochen. Als Ergebnis kam vor allem heraus,
dass das Framework ausschließlich die Kommunikation über unverschlüsselte Kanäle
anbietet. Dies ist jedoch in sehr vielen IoT-Anwendungen eine Verletzung der
IT-Sicherheit. Besonders bei unserem Beispiel mit der Sicherheitskamera konnten
wir zeigen, dass aufgrund der Tatsache, dass keine geeignete Verschlüsselung
verwendet wird und Bildaufnahmen der Kamera mit einfachsten Mitteln ausgespäht
werden können. Auch schützt das Framework nicht vor Fälschung der Daten, sodass
die Bildaufnahmen manipuliert zum Broker gelangen können.
