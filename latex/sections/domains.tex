\section{IoT Anwendungsbereiche}
Aufgrund der allgemeinen Digitalisierung vieler Abläufe und Branchen werden IoT-Geräte umfangreich eingesetzt,
in dem folgenden Kapitel sollen einige Anwendungsbereiche der IoT vorgestellt werden. 

\subsection{Umweltbeobachtung}
Bei der Umweltbeobachtung werden bestimmte Bereiche der Umwelt auf bestimmte ökologische Parameter überwacht. 
So werden bspw. die Luft-, Wasser- oder Bodenqualität von bestimmten Gebieten oder Standorten 
untersucht, um potentielle Umweltprobleme oder -gefahren frühzeitig feststellen zu können. \\

Ein detailliertes Beispiel wäre die Erdbebenüberwachung, bei der IoT-Geräte mit seismischen Sensoren
ausgestattet werden, um frühzeitig Erdbeben zu erkennen. Im Falle eines Erdbebens können die Geräte
weitere System benachrichtigen, welche daraufhin Versorgungseinrichtungen oder  Datenzentren
abschalten, sowie Menschen in kritischen Wohngebieten benachrichtigen können.

\subsection{Gesundheitswesen}
Ärzte benötigen oftmals stetige Information zum körperlichen Zustand eines Patienten, vor allem um
herauszufinden, ob eine bestimmte Behandlung zum gewünschten Erfolg führt. IoT-Geräte, im speziellen
kleine medizinische Implantate, können Ärzte dabei unterstützen bestimmte Gesundheitswerte auszulesen und 
zu versenden bzw. selbständig auszuwerten. Ärzte können dadurch, ohne das physische auftreten der Patienten,
Untersuchungen und Diagnosen durchführen, was sowohl dem Arzt und Patient Zeit und Aufwand einsparen kann.

\subsection{Industrie}
In der Produktion von Gütern finden IoT-Geräte sehr unterschiedliche Verwendungszwecke. Mithilfe der IoT können
bspw. gesamte Lieferketten optimiert werden, indem IoT-Geräte bestimmte Teilprozesse (Produktion, Transport, etc.)
überwachen und in bestimmten Situation zielgerichtete Maßnahmen einleiten. Außerdem werden mithilfe von IoT-Geräten eine 
Vielzahl von Daten ausgehoben, welche in wachsenden Anwendungsbereichen wie Big Data Analysen oder Künstlicher Intelligenz 
benötigt werden.

\subsection{Wearables}
Bei Wearables handelt es sich um technische Geräte, welche direkt am Körper getragen werden und mit verschiedenen 
Sensoren ausgestattet sind. Die bekanntesten Beispiele für Wearabeles sind Smartwachtes und Fitnessarmbänder und 
werden für unterschiedliche Anwendungsbereiche wie bspw. Gesundheitswesen, Sport oder Entertainment eingesetzt.


\subsection{Spielzeuge}
IoT-Geräte in Form von Spielzeugen sind oftsmals mit Lautsprecher, Mikrofonen und wieteren Komponenten ausgesattet
und mit dem Internet verbinden. Funktionalitäten, welche klassische Spielzeuge nicht haben, wäre bspw. eine integrierte
Spracherkennung, bei dir die Aussagen der Kinder aufgenommen, über ein weiteres System ausgewertet und das Spielzeug
demensprechend reagiert. Aufgrund der Verbindung zum Internet stehen den Spielzeugen eine Vielzahl von neuen Möglichkeiten
zur Verfügung.