\section{Anwendungsbereiche}
Aufgrund der allgemeinen Digitalisierung werden heutzutage IoT-Geräte in 
vielen Branchen und Abläufen umfangreich eingesetzt. In dem folgenden Kapitel 
sollen einige Anwendungsbereiche der IoT vorgestellt werden. 

\paragraph{Umweltbeobachtung.}
Bei der Umweltbeobachtung werden bestimmte Bereiche der Umwelt auf bestimmte
ökologische Parameter überwacht.  So werden bspw. die Luft-, Wasser- oder
Bodenqualität von bestimmten Gebieten oder Standorten untersucht, um potentielle
Umweltprobleme oder -gefahren frühzeitig feststellen zu können.  Ein
detailliertes Beispiel wäre die Erdbebenüberwachung, bei der IoT-Geräte mit
seismischen Sensoren ausgestattet werden, um frühzeitig Erdbeben zu erkennen. Im
Falle eines Erdbebens können die Geräte weitere System benachrichtigen, welche
daraufhin Versorgungseinrichtungen oder  Datenzentren abschalten, sowie Menschen
in kritischen Wohngebieten benachrichtigen können \cite{paper}.

\paragraph{Gesundheitswesen.}
Ärzte benötigen oftmals stetige Information zum kör\-per\-lich\-en Zustand eines
Patienten, um herauszufinden, ob eine bestimmte Behandlung zum
gewünschten Erfolg führt. IoT-Geräte, im speziellen kleine medizinische
Implantate, können Ärzte dabei unterstützen bestimmte Gesundheitswerte
auszulesen bzw.  selbständig auszuwerten. Ärzte können dadurch einfacher
Untersuchungen und Diagnosen durchführen \cite{paper}.

\paragraph{Industrie.}
In der Produktion von Gütern finden IoT-Geräte sehr unterschiedliche
Verwendungszwecke. Mithilfe der IoT können bspw. gesamte Lieferketten optimiert
werden, indem IoT-Geräte bestimmte Teilprozesse (Produktion, Transport, etc.)
überwachen und in bestimmten Situation zielgerichtete Maßnahmen einleiten.
Außerdem werden mithilfe von IoT-Geräten eine Vielzahl von Daten ausgehoben,
welche in wachsenden Anwendungsbereichen wie Big Data Analysen oder Künstlicher
Intelligenz benötigt werden \cite{paper}.

\paragraph{Wearables.}
Bei Wearables handelt es sich um technische Geräte, welche direkt am Körper
getragen werden und mit verschiedenen Sensoren ausgestattet sind. Die
bekanntesten Beispiele für Wearabeles sind Smart-Watches und Fitnessarmbänder
und werden für unterschiedliche Anwendungsbereiche wie bspw. Gesundheitswesen,
Sport oder Entertainment eingesetzt \cite{paper}.

\paragraph{Spielzeuge.}
IoT-Geräte in Form von Spielzeugen sind oftmals mit Lautsprecher, Mikrofonen und
weiteren technischen Komponenten ausgestattet.  Funktionalitäten, welche
klassische Spielzeuge nicht haben, wären bspw. eine integrierte Spracherkennung,
bei dir die Aussagen der Kinder aufgenommen, über ein weiteres System
ausgewertet werden und das Spielzeug dementsprechend reagiert. Aufgrund der
Verbindung zum Internet stehen den Spielzeugen dadurch eine Vielzahl von neuen
Möglichkeiten zur Verfügung \cite{paper}.
