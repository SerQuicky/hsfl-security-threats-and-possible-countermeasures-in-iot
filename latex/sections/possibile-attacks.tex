\section{Angriffsmöglichkeiten}
Wie in fast jedem anderen System müssen vor allem IoT-Anwendungen besonders
bezüglich ihrer Sicherheit betrachtet werden. Die Anwendungsentwicklung
innerhalb des IoT ist geprägt von vielseitig komplexen Problemen beim Etablieren
der IT-Sicherheit. Grundsätzlich können Sicherheitsrisiken in drei Bereiche
untergliedert werden, die hier kurz erläutert werden sollen.

\paragraph{Heterogenität.}
Dieser Bereich behandelt Probleme, die durch die Vielzahl an
Anwendungsmöglichkeiten in dem IoT entstehen. Dazu zählen unter anderem
die verbauten Hardware-Komponenten, Protokolle für die Kommunikation unter
Geräten und Variationen zwischen Geräten und ihrer Rechenleistung.

\paragraph{Ressourcenbeschränkung.}
Nicht jedes Gerät beinhaltet einen leistungsstarken Rechner. Oft stößt man bei
der Entwicklung von IoT-Geräten auf das Problem, dass nicht genug Speicher oder
Rechenleistung vorhanden ist. Nicht nur bei der Implementation der eigentlichen
Anwendung macht dies Probleme. Bereits etablierte Sicherheitsalgorithmen können
aufgrund fehlender Leistung nicht verwendet werden, da sie sich als
unpraktikabel in dem IoT herausstellen. Sicherheitsziele können also aufgrund
nicht ausreichender Hardware nicht erreicht werden. Ein Beispiel hierfür ist das
Arbeiten mit einer Cloud. Sicherheitsziele ist hier unter anderem die
Privatsphäre des Benutzers. Die üblichen kryptographischen Algorithmen, z.B. die
attribut-basierte Verschlüsselungen und Pairings, stellen sich als sehr
unpraktikabel heraus, da diese zu viel Rechenleistung für die Berechnungen von
bilinearen Abbildungen benötigen. Eine vielversprechende Lösung ist das
Auslagern rechenintensiver Schritte an einem leistungsstarken Server
\cite{phoabe}.

\paragraph{Dynamische Netzwerktopologie.}
