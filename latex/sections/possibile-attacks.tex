\section{Angriffsmöglichkeiten}\label{sec:possible-attacks}
Wie in fast jedem anderen System müssen vor allem IoT-Anwendungen besonders
bezüglich ihrer Sicherheit betrachtet werden. Die Anwendungsentwicklung
innerhalb des IoT ist geprägt von vielseitig komplexen Problemen beim Etablieren
der IT-Sicherheit. Grundsätzlich können Sicherheitsrisiken in drei Bereichen
auftreten, die hier kurz erläutert werden sollen \cite{paper}.

\subsection{Herausforderungen bei mobilen Geräten}
\paragraph{Heterogenität}
Dieser Bereich behandelt Probleme, die durch die Vielzahl an
Anwendungsmöglichkeiten in dem IoT entstehen. Dazu zählen unter anderem
die verbauten Hardware-Komponenten, Protokolle für die Kommunikation unter
Geräten und Variationen zwischen Geräten und ihrer Rechenleistung \cite{paper}.

\paragraph{Ressourcenbeschränkung}
Nicht jedes Gerät beinhaltet einen leistungsstarken Rechner. Oft stößt man bei
der Entwicklung von IoT-Geräten auf das Problem, dass nicht genug Speicher oder
Rechenleistung vorhanden ist. Nicht nur bei der Implementation der eigentlichen
Anwendung macht dies Probleme. Bereits etablierte Sicherheitsalgorithmen können
aufgrund fehlender Leistung nicht verwendet werden, da sie sich als
unpraktikabel in dem IoT herausstellen. Sicherheitsziele können also aufgrund
nicht ausreichender Hardware nicht erreicht werden \cite{paper}. Ein Beispiel
hierfür ist das Arbeiten mit einer Cloud. Sicherheitsziele sind hier unter
anderem die Privatsphäre des Benutzers. Die üblichen kryptographischen
Algorithmen, z.B. die attribut-basierte Verschlüsselungen und Pairings, stellen
sich als sehr unpraktikabel heraus, da diese zu viel Rechenleistung für die
Berechnungen von bilinearen Abbildungen benötigen. Eine vielversprechende Lösung
ist das Auslagern rechenintensiver Schritte an einen leistungsstarken Server
\cite{phoabe}.

\paragraph{Dynamische Netzwerktopologie}
Gerade bei mobilen Geräten innerhalb des IoT haben wir mit ``losen'' Verbindungen
zu tun. Zum Beispiel ist es möglich, dass ein Smartphone mit diversen
WLAN-Hotspots interagiert und damit keine feste Position innerhalb eines
Netzwerkes besitzt \cite{paper}.

\subsection{Definition von Sicherheit}\label{sec:def-security}
Damit die Sicherheit trotz der oben erwähnten Herausforderungen ge\-währ\-lei\-stet
werden kann, genügt es nicht, dass allseits bekannte CIA-Dreieck zu betrachten.
Laut \cite{paper} sind Vertraulichkeit, Integrität und Verfügbarkeit nicht die
einzigen Voraussetzungen, um Sicherheit für IoT-Geräte zu definieren.
Stattdessen müssen individuelle Anforderungen für jeden Anwendungsbereich
definiert werden.

Komplexe Anlagen besitzen zum Beispiel verschiedenste Sensoren und Aktoren, um
beispielsweise Verfahrenstechniken anzuwenden und diese zu über\-wachen. Diese
Sensoren werden häufig beim Implementieren von IT-Sich\-er\-heit vernachlässigt,
nachdem sie verbaut wurden. Mögliche Sicherheitsanforderungen für
diesen Bereich sind Vertraulichkeit, Authentifizierung,
Nachrichtenauthentizität, Integrität, Verfügbarkeit, Zuverlässigkeit, Frische
der Daten und Fälschungsdetektierung \cite{paper}.

Andere Bereiche des IoT benötigen jedoch andere Anforderungen an die
IT-Sicherheit. Im Gesundheitswesen wird die Privatsphäre des Patienten eine
größere Rolle spielen, als bei technischen Anlagen, sodass die Anforderungen
abweichen. Diese sind hier unter anderem Verfügbarkeit, Zuverlässigkeit,
Authentifizierung, Integrität, Vertraulichkeit, Privatsphäre,
Nachrichtenauthentizität, Fälschungsdetektierung, Verantwortlichkeit und
Nichabstreitbarkeit \cite{paper}.

\subsection{Sicherheitsrisiken pro Domäne}
Im vorherigen Abschnitt wurde auf verschiedene IoT-Domänen und ihren
Sicherheitsanforderungen eingegangen. Kurz zusammengefasst ist jede Domäne
unterschiedlich zu behandeln, wenn es um Sicherheit geht. Angreifer, die
Schwachstellen in solchen Systemen finden, haben unter Umständen Zugriff auf
eine große Menge von sensiblen Daten. Mögliche Angriffe können auf Grundlage der
Sicherheitsanforderungen aus Abschnitt \ref{sec:def-security} abgeleitet werden
\cite{paper} und werden in Tabelle \ref{tab:security-threats} dargestellt.

\begin{table*}[t]
  \centering
  \label{tab:security-threats}
  \begin{tabularx}{\textwidth}{lX}
    \textbf{Anwendungsdomäne} & \textbf{Mögliche Angriffe}\\
    \hline
    Umgebungsüberwachung & DoS/DDoS, MitM, Node capture, Sinkhole, Hello flood,
    Traffic analysis, Hacking, Side channel, Sybil, Selective forwarding, Back
    hole, Tampering, Wormhole, Masquerading \\

    Gesundheitswesen & DoS/DDoS, MitM, Malicious code, Spoofing, Hacking,
    Tampering, Eavesdropping, Hijacking, Replay, Backdoof, Tag tracking, Tag
    cloning, Identity theft, Masquerading, Node capture, Side channel \\

    Feuerwehr & DoS/DDoS, MitM, Sybil, Hello flood, Node capture, Sinkhole,
    Black hole, Selective forwarding, Hacking, Tampering, Malicious code,
    Hijacking, Side channel, Traffic analysis, GPS jamming \\

    Herstellung & DoS/DDoS, MitM, Masquerading, Backdoor, Identity theft,
    Replay, Hijacking, Hacking, Eavesdropping, Selective forwarding, Back hole,
    Sinkhole, Node capture, Sybil, Spoofing, Traffic analysis, Side channel,
    Tampering, Wormhole, Malicious code, Wormhole, Economic espionage \\

    Wearables & DoS/DDoS, Eavesdropping, MitM, Malicious code, Identity theft,
    Hacking, Backdoor, Hijacking, inappropriate network configuration \\

    Spielzeuge & DoS/DDoS, Eavesdropping, MitM, Identity theft, Hijacking,
    Hacking, Backdoor, Masquerading, Spoofing, Malicious code, inappropriate
    network configuration
  \end{tabularx}
  \caption{Sicherheitsrisiken pro Domäne \cite{paper}}
\end{table*}
