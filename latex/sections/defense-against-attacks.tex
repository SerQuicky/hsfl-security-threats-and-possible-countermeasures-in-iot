\section{Verteidigung gegen Angriffe}

\paragraph{Gegenmaßnahmen gegen unsichere Web-Interfaces und Netzwerkdienste.}
Damit ein Fluten von Authentifizierungsanfragen verhindert werden kann, können
Standardports geblockt werden. Für SSH ist dies beispielsweise der Port 22 und
für Telnet der Port 23 bzw. 2323. Grundsätzlich sollten nur die Ports
geöffnet werden, die wirklich benötigt werden. Auch das \textit{Universial Plug
and Play Protocol} (UPnP) sollte deaktiviert werden. Außerdem werden häufig
Standardeinstellungen wie Benutzernamen und Passwörter unverändert übernommen.
Dies ist offensichtlich ein großes Problem, denn ein Angreifer wird diese
Information ohne großen Aufwand herausfinden können. Man sollte also darauf
achten, individuelle und starke Passwörter zu verwenden. Auch die Software
selbst sollte regelmäßig Updates eingespielt bekommen, denn häufig werden
Schwachstellen erst nach einem Release festgestellt und Patches
veröffentlicht \cite{paper}.

\paragraph{Gegenmaßnahmen für Routing-Protokolle.}
Da IoT-Geräte sehr eingeschränkter Natur sind, ist das Schützen von
IoT-Netzwerken eine echte Herausforderung \cite{patel2016}. Sinkhole-Angriffe
können laut \cite{paper} verhindert werden, indem robuste
Authentifikation-Schemata, geografisches Routing und systematisches Rerouting
verwendet werden.  Geografisches Routing bedeutet, dass Datenpakete abhängig der
Position von Knotenpunkten und der Zieladresse innerhalb des Netzwerks
umgeleitet werden.  Die Chancen, dass Forwarding- und Black-Hole-Angriffe
durchführbar sind, können mithilfe von Source-Routing reduziert werden.
Verwendet man Multipath-Routing, dann können Selective-Forwarding-Angriffe
abgewehrt werden. Hierbei werden verschiedene Pfade innerhalb des Netzwerks
verwendet. Hello-Flood-Angriffe können durch bidirektionale Authentifikation
verhindert werden.  Wormhole-Angriffe können ebenfalls durch geografisches
Routing erschwert durchführbar gemacht werden. Aber auch durch physische
Überwachung oder Source-Rou\-ting können Angriffe dieser Art verhindert werden.
Gegenmaßnamen gegen Sybil-Angriffe ist die Evaluierung der Einzigartigkeit von
Geräten im Netzwerk.  Dies bedeutet lediglich, dass jedes Gerät eine eindeutige
Identifikationsnummer besitzen muss. Auch mithilfe von sogenannten
Random-Key-Pre-Dis\-tri\-bu\-tion-Schemata \cite{chan2003} können Sybil-Angriffe
verhindert werden.

\paragraph{Gegenmaßnahmen für Informationstransport}
Bevor ein Smart-Device einen Datenaustausch beginnt, sollte sich dieses
Gerät vorerst authentifizieren. Ebenfalls sollte eine beidseitige
Authentifizierung implementiert werden, damit beide Parteien verifizieren
können, ob es sich um den richtigen Endpunkt handelt und sie mit dem gewünschten
Ziel kommunizieren. Die Idee hierbei ist, dass zum Beispiel digitale Signaturen
(SHA, ECDSA) bzw. symmetrische äquivalente (HMAC) ausgetauscht werden. Dies
dient vor allem für die Einhaltung der Integrität der Daten. Ebenfalls können
Signaturen verwendet werden, um einen Secure-Boot zu implementieren. Dies hilft
dabei, dass das IoT-Gerät nur Codes ausführt, die vertrauenswürdig, also
signiert sind. Vor allem wird dadurch verhindert, dass zum Beispiel die Firmware
des Geräts überspielt wird. Um die Angriffe abzuwehren, die die Vertraulichkeit
betreffen, werden Informationen über einen sicheren, verschlüsselten Kanal
übertragen. Dies betrifft auch die Kommunikation mit externen Diensten, wie
einer Cloud.
