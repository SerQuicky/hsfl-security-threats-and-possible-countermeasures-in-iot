\section{Verteidigung gegen Angriffe}\label{sec:defense-against-attacks}
Um Angriffen, die in Tabelle \ref{tab:security-threats} genannt wurden,
entgegenzuwirken, sollen nun einige Maßnahmen für die Abwehr besprochen werden.

\paragraph{Gegenmaßnahmen gegen unsichere Web-Interfaces und Netzwerkdienste.}
Damit ein Fluten von Authentifizierungsanfragen verhindert werden kann, können
Standardports geblockt werden. Für SSH ist dies beispielsweise der Port 22 und
für Telnet der Port 23 bzw. 2323. Grundsätzlich sollten nur die Ports geöffnet
werden, die wirklich benötigt werden. Auch das \textit{Universial Plug and Play
Protocol} (UPnP) sollte deaktiviert werden. Außerdem werden häufig
Standardeinstellungen wie Benutzernamen und Passwörter unverändert übernommen.
Dies ist offensichtlich ein großes Problem, denn ein Angreifer wird diese
Information ohne großen Aufwand herausfinden können. Man sollte also darauf
achten, individuelle und starke Passwörter zu verwenden. Auch die Software
selbst sollte regelmäßig Updates eingespielt bekommen, denn häufig werden
Schwachstellen erst nach einem Release festgestellt und Patches veröffentlicht
\cite{paper}.

\paragraph{Gegenmaßnahmen gegen Routing-Protokolle.}
Da IoT-Geräte sehr eingeschränkter Natur sind, ist das Schützen von
IoT-Netzwerken eine echte Herausforderung \cite{patel2016}. Sinkhole-Angriffe
können laut \cite{paper} verhindert werden, indem robuste
Authentifikation-Schemata, geografisches Routing und systematisches Rerouting
verwendet werden.  Geografisches Routing bedeutet, dass Datenpakete abhängig der
Position von Knotenpunkten und der Zieladresse innerhalb des Netzwerks
umgeleitet werden.  Die Chancen, dass Forwarding- und Black-Hole-Angriffe
durchführbar sind, können mithilfe von Source-Routing reduziert werden.
Verwendet man Multipath-Routing, dann können Selective-Forwarding-Angriffe
abgewehrt werden. Hierbei werden verschiedene Pfade innerhalb des Netzwerks
verwendet. Hello-Flood-Angriffe können durch bidirektionale Authentifikation
verhindert werden.  Wormhole-Angriffe können ebenfalls durch geografisches
Routing erschwert durchführbar gemacht werden. Aber auch durch physische
Überwachung oder Source-Rou\-ting können Angriffe dieser Art verhindert werden.
Gegenmaßnamen gegen Sybil-Angriffe ist die Evaluierung der Einzigartigkeit von
Geräten im Netzwerk.  Dies bedeutet lediglich, dass jedes Gerät eine eindeutige
Identifikationsnummer besitzen muss. Auch mithilfe von sogenannten
Random-Key-Pre-Dis\-tri\-bu\-tion-Schemata \cite{chan2003} können Sybil-Angriffe
verhindert werden.

\paragraph{Gegenmaßnahmen gegen unerlaubten Informationstransport.}
Bevor ein Smart-De\-vice einen Datenaustausch beginnt, sollte sich dieses Gerät
vorerst authentifizieren. Ebenfalls sollte eine beidseitige Authentifizierung
implementiert werden, damit beide Parteien verifizieren können, ob es sich um
den richtigen Endpunkt handelt und sie mit dem gewünschten Ziel kommunizieren.
Die Idee hierbei ist, dass zum Beispiel digitale Signaturen (SHA, ECDSA) bzw.
symmetrische äquivalente (HMAC) ausgetauscht werden. Dies dient vor allem für
die Einhaltung der Integrität der Daten. Ebenfalls können Signaturen verwendet
werden, um einen Secure-Boot zu implementieren. Dies hilft dabei, dass das
IoT-Gerät nur Codes ausführt, die vertrauenswürdig, also signiert sind. Vor
allem wird dadurch verhindert, dass zum Beispiel die Firmware des Geräts
überspielt wird. Um die Angriffe abzuwehren, die die Vertraulichkeit betreffen,
werden Informationen über einen sicheren, verschlüsselten Kanal übertragen. Dies
betrifft auch die Kommunikation mit externen Diensten, wie einer Cloud
\cite{paper}.

\paragraph{Gegenmaßnahmen gegen physische Angriffe.}
Um zum Beispiel Node-Capturing zu verhindern, müssen Geräte physisch versteckt
werden, um den Zugang zu erschweren. Außerdem können die Geräte so gebaut
werden, sodass diese nur sehr schwer auseinanderbaubar sind. USB-Ports sollten
zudem auch geschützt werden, damit ein Angreifer keine Schadsoftware über diese
einspielen kann. Schutzmaßnahmen gegen das Manipulieren von Daten können durch
regelmäßige Änderung der Schlüssel eingeführt werden. Side-Channel-Angriffe
können durch spezielle, resistente Chipsätze verhindert werden.  Das Messen von
elektromagnetische Strahlung muss beim Entwickeln von IoT-Geräten ebenfalls
berücksichtigt werden. Das Verschleiern von Informationen auf diesem Wege muss
also ebenfalls implementiert werden, damit Angreifer nicht mithilfe von
elektrischen Signalen Zugriff bekommen \cite{paper}.

\paragraph{Gegenmaßnahmen gegen GPS-Jamming.}
Um GPS-Jamming-Angriffe abzuwehren, können Kerbfilter eingesetzt werden
\cite{paper,borio2012}.

\paragraph{Gegenmaßnahmen gegen Tag-Tracking und -Cloning.}
Eine Schutzmaßnahme gegen das Tag-Tracking ist, dass die Tags gegenüber einem
Angreifer zufällig aussehen und eine gleiche Verteilung besitzen, damit dieser
keine Schlüsse aus dem Aufbau der Tags ziehen kann. Um Tag-Cloning zu
verhindern, muss sichergestellt werden, dass die eigentlichen Informationen
unter dem Tag nicht zugänglich sind, um einen neuen validen Tag zu erzeugen
\cite{paper}.

\paragraph{Gegenmaßnahmen gegen falsche Netwerkkonfiguration.}
Die Schutzmaßnahme ist hier sozialer Natur. Ein Schulen der Benutzer über die
Wichtigkeit der IT-Sicherheit ist unabdingbar. Es müssen starke Regeln für
Passwörter verwendet und sicherheitsrelevantes Logging aktiviert werden
\cite{paper}.
