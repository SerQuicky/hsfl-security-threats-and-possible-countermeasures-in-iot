\section{Einleitung}
\textit{Internet of Things} (IoT) bezeichnet die Vernetzung von Geräten
über das Internet.  Diese werden oftmals, aufgrund der Verbindung zum Internet,
auch als \textit{intelligente Geräte (engl. smart devices)} bezeichnet. Die
Endgeräte können dabei sehr unterschiedlich sein, so reichen diese von
einfachen Messgeräten bis hin zu komplexen medizinischen Implantaten
\cite{paper}.

Aufgrund der Anbindung der Geräte an das Internet, ist die Kommunikationen mit
diesen einfacher und Daten können leichter extrahiert und übermittelt werden.
Diese Prozesse können zudem automatisiert werden, wodurch keine weitere
menschliche Interaktion benötigt wird und Zeit bzw. Kosten eingespart werden
können. Dadurch ist nicht verwunderlich, dass die Anzahl an IoT-Geräten in den
letzten Jahren stark zugenommen hat. So sollen bis Ende 2020 über 25 Milliarden
Geräte über das Internet vernetzt sein \cite{paper}.

Die beeindruckende Anzahl an IoT-Geräten zeigt, dass die Technologie sich
heutzutage etabliert hat, doch sollten hierbei nicht die Sicherheitsrisiken bei
der Verwendung von IoT-Geräten außer Acht gelassen werden. Abhängig vom
Anwendungsbereich der Geräte können diese für essentielle Funtkionalitäten eines
System zuständig sein oder mit sehr sensiblen Daten arbeiten.  Geräte, welche
nicht ausreichend geschützt sind, können durch Angriffe bspw.  heruntergefahren
werden, was in Abhängigkeit zum Anwendungsbereich zu erheblichen Schäden führen
kann \cite{paper}.

Im Rahmen diese Ausarbeitung werden die Anwendungsbereiche, Sicherheitsrisiken
und Schutzmechanismen von IoT-Geräten vorgestellt und anhand eines praktischen
Beispiels demonstriert.
